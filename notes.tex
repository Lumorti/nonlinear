\documentclass{article}

\usepackage{amsmath}
\usepackage{amssymb}

\setlength{\parskip}{1em}
\setlength{\parindent}{0em}
\setcounter{MaxMatrixCols}{20}

\begin{document}

For a given dimension $d$, number of sets $s$ and a series of measurements $M_i^k \in \mathbb{C}^d \otimes \mathbb{C}^d$ our problem is as follows:

\begin{align}
	\text{min} \sum_{i,j}^s \sum_{k,l}^d -\sqrt{1-\text{tr}(M_k^i M_l^j)} 
\end{align}

subject to:

\begin{align}
	M_k^i &\succeq 0 \\
	\sum_k^d M_k^i &= \mathbb{I} \\
	\text{tr}(M_k^i) &= 1 \\
	(M_k^i)^2 &= M_k^i
\end{align}

Need to define three functions for this method: the objective function $f(x)$, the constraint function $g(x)$ and the function $X(x)$ converting between the vector form $x$ and the matrix form such that $X(x)$ is positive semidefinite. These functions do not need to be linear.

In our case $X(x)$ takes the form:

\begin{equation}
	X(x) = \sum_a^n x_a A_a + B
\end{equation}

such that it turns a vector $x \in \mathbb{R}^n$ containing the unique elements of the measurements into a matrix $X(x) \in \mathbb{R}^p \otimes \mathbb{R}^p$ which contains the real components on the block diagonals and the imaginary components on the off-diagonals, with $p=2d^2n$. For instance, in the $d=2$, $n=2$ case, if each $M_i^k$ = $R_i^k + iI_i^k$:

\begin{equation}
	X(x) = 
	\begin{pmatrix}
		R_1^1 & 0 & 0 & 0 & I_1^1 & 0 & 0 & 0 \\
		0 & R_2^1 & 0 & 0 & 0 & I_2^1 & 0 & 0 \\
		0 & 0 & R_1^2 & 0 & 0 & 0 & I_1^2 & 0 \\
		0 & 0 & 0 & R_2^2 & 0 & 0 & 0 & I_2^2 \\
		I_1^1 & 0 & 0 & 0 & R_1^1 & 0 & 0 & 0 \\
		0 & I_2^1 & 0 & 0 & 0 & R_2^1 & 0 & 0 \\
		0 & 0 & I_1^2 & 0 & 0 & 0 & R_1^2 & 0 \\
		0 & 0 & 0 & I_2^2 & 0 & 0 & 0 & R_2^2
	\end{pmatrix}
\end{equation}

Many of these elements are defined in relation to others such that the vector $x$ contains as little information as needed, whilst also forcing the submatrices of any $X(x)$ to satisfy the identity and trace constraints. This also has the benefit of meaning in our case $g(x)$ is only used for the projective constraint.

Splitting the objective function into real and imaginary components and using the identity $\text{tr}(M_k^i M_l^j) = M_k^i \cdot M_l^j$:

\begin{equation}
	f(x) = \sum_{i,j}^s \sum_{k,l}^d -\sqrt{1 - R_k^i \cdot R_l^j + I_k^i \cdot I_l^j}
\end{equation}

Now defining the extraction matrices $C_k^i$, $D_k^i$, $E_k^i$ and $F_k^i$ such that $C_k^i X D_k^i = R_k^i$ and $E_k^i X F_k^i = I_k^i$ with the notation that $X = X(x)$:

\begin{equation}
	f(x) = \sum_{i,j}^s \sum_{k,l}^d -\sqrt{1 - C_k^i X D_k^i \cdot C_l^j X D_l^j + E_k^i X F_k^i \cdot E_l^j X F_l^j} 
\end{equation}

Thus the full expression in terms of the components of $x$: 

\begin{align}
	f(x) = \sum_{i,j}^s \sum_{k,l}^d -\bigg(1 - C_k^i (\sum_a^n x_a A_a + B) D_k^i \cdot C_l^j (\sum_a^n x_a A_a + B) D_l^j \\
	+ E_k^i (\sum_a^n x_a A_a + B) F_k^i \cdot E_l^j (\sum_a^n x_a A_a + B) F_l^j\bigg)^{\frac{1}{2}} 
\end{align}

Now taking the first derivative of this, letting $d_k^i$ be the value inside the above square root:

\begin{align}
	\frac{\partial f(x)}{\partial x_b} = \sum_{i,j}^s \sum_{k,l}^d -\frac{1}{2}(d_k^i)^{-\frac{1}{2}}(-C_k^i A_b D_k^i \cdot C_l^j X D_l^j - C_k^i X D_k^i \cdot C_l^j A_b D_l^j \\
	+ E_k^i A_b F_k^i \cdot E_l^j X F_l^j + E_k^i X F_k^i \cdot E_l^j A_b F_l^j) 
\end{align}

Then the second derivative, remembering that $d_k^i$ has a dependence on $x$:

\begin{align}
	\frac{\partial^2 f(x)}{\partial x_b \partial x_c} = \sum_{i,j}^s \sum_{k,l}^d \frac{1}{4}(d_k^i)^{-\frac{3}{2}}(-C_k^i A_b D_k^i \cdot C_l^j X D_l^j - C_k^i X D_k^i \cdot C_l^j A_b D_l^j \\
	+ E_k^i A_b F_k^i \cdot E_l^j X F_l^j + E_k^i X F_k^i \cdot E_l^j A_b F_l^j) \\
	(-C_k^i A_c D_k^i \cdot C_l^j X D_l^j - C_k^i X D_k^i \cdot C_l^j A_c D_l^j \\
	+ E_k^i A_c F_k^i \cdot E_l^j X F_l^j + E_k^i X F_k^i \cdot E_l^j A_c F_l^j) \\
	-\frac{1}{2}(d_k^i)^{-\frac{1}{2}}(-C_k^i A_b D_k^i \cdot C_l^j A_c D_l^j - C_k^i A_c D_k^i \cdot C_l^j A_b D_l^j \\
	+ E_k^i A_b F_k^i \cdot E_l^j A_c F_l^j + E_k^i A_c F_k^i \cdot E_l^j A_b F_l^j)
\end{align}

Note that in the above expressions many of these terms are zero, for instance $C_k^i A_b D_k^i \cdot C_l^j A_b D_l^j = 0$ since each A matrix will only have non-zero components for a single measurement, so one of those two extractions must result in a zero matrix.

Computationally many things here are different, such that $X$ is only ever calculated once per iteration and the extraction matrices are unneeded, instead the submatrices are extracted directly from the cached $X$ using Eigen's ``X.block()'' routine, which claims $O(0)$ scaling when compiled with optimisations.

Putting these all together we can construct the Lagrangian and it's first/second derivatives, which are then used as in the paper.



\end{document}
